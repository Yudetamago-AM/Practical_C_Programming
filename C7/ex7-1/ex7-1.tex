\documentclass[a4paper,11pt]{jsarticle}

% 数式
\usepackage{amsmath,amsfonts}
\usepackage{bm}
% 画像
\usepackage[dvipdfmx]{graphicx}
% 余白設定
\usepackage[top=30truemm,bottom=30truemm,left=25truemm,right=25truemm]{geometry}

\begin{document}

\title{ex7-1:ヤードポンド法→メートル法変換プログラム\\{\Large 予備仕様書}}
\author{Yudetamago-AM}
\date{2022/7/24}
\maketitle

\section*{概要}

このプログラムは,しばしば人を困らせるヤードポンド法をメートル法に変換するプログラムです.
このプログラムは,ヤードポンド法の単位のうち,以下のような主要な物をサポートします.
また,同時にその単位の略称,単位量あたりのメートル法での単位量および単位を以下のように定めます.
というのも,ヤードポンド法では,同じ単位でも国・地域や考え方によりそれが指す量が変わることがあるためです.

\begin{table}[h]
  \caption{単位}
  \centering
  \begin{tabular}{llll}
    \hline
    \multicolumn{2}{l}{ヤードポンド法} & \multicolumn{2}{l}{メートル法}\\
    単位 & 略称 & 単位量 & 単位\\
    \hline \hline
    マイル & mi & 1.6093 & $ km $\\
    ヤード & yd & 91.440 & $ cm $\\
    フィート & ft & 30.48 & $ cm $\\
    インチ & in & 2.5400 & $ cm $\\
    エーカー & ac & 4046.9 & $ m^2 $\\
    バレル & bbl & 159.00 & $ L $\\
    ガロン & gal & 3.7854 & $ L $\\
    ポンド & lb & 453.59237 & $ g $\\
    オンス & oz & 28.350 & $ g $\\
    \hline
  \end{tabular}
\end{table}





\end{document}